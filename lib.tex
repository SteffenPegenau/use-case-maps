\documentclass[tikz,border=10pt]{standalone}
\usepackage{verbatim}
\usepackage{calc}
\usepackage[utf8]{inputenc}         % Umlaute etc.
\usepackage[T1]{fontenc}            % T1-kodierte Fonts
\usepackage{ae,aecompl}             % Kodierung für PDF
\usepackage{ngerman}                % Deutsche Trennungen,
\usepackage{intcalc}
\usepackage{amsmath}
\usepackage{tikz}

\usetikzlibrary{positioning,shapes, shapes.geometric}

\pgfmathsetbasenumberlength{4}
\def\ns{\nodepart{second}}
\def\ns{\nodepart{third}}

\tikzset{
  up/.style = {
    fill=red
  },
  struc/.style = {
    font=\scriptsize\sffamily, 
    rectangle split,
    rectangle split parts=3,
    rectangle split part fill={\topFillColor,\middleFillColor,\bottomFillColor},
    %rectangle split part draw={red,white,blue},
    minimum height = 2.5cm,
    minimum width=3cm,
    inner sep=1pt,
    text width= 3cm,
    align=center,
    inner sep=0.15cm,
    %font=\scriptsize\bfseries,
    draw,
    fill=#1
    },
  struc/.default=none
}


% Settings


\newenvironment
  {storyMap}
  {
    \newcounter{LastStep}
    \setcounter{LastStep}{0}
    \newcounter{StepNumber}
    \newcommand{\story}[2]
    {
      
      \addtocounter{StepNumber}{1}
      \node [struc,x=0cm,y=-1cm] at ({\intcalcMod{\theStepNumber-1}{\boxesPerRow} * 4.5cm},{floor(\theLastStep / \boxesPerRow) - 1)*-4cm}) (step\theStepNumber) {Schritt \theStepNumber \nodepart{second} ##2 \nodepart{third} ##1 }  ;
      
      \ifnum\theStepNumber=1
      \else
	\ifnum\intcalcMod{\theStepNumber-1}{\boxesPerRow} = 0
	  \draw [->, very thick] (step\theLastStep) -- ++(2.5cm,0) -- ++(0,-2) -|  (step\theStepNumber);
	\else
	  \draw [->, very thick] (step\theLastStep) -- (step\theStepNumber);
	\fi
      \fi
       \addtocounter{LastStep}{1}
      ;
      
    }
    \begin{tikzpicture}[node distance=2cm]
    
  }
  {
    \end{tikzpicture}
  }

